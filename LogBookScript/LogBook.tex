\documentclass{article}%
\usepackage[T1]{fontenc}%
\usepackage[utf8]{inputenc}%
\usepackage{lmodern}%
\usepackage{textcomp}%
\usepackage{lastpage}%
%
%
%
\begin{document}%
\normalsize%
\section{Logbook de Antoine}%
\textbf{Date : }%
\textbf{2018{-}02{-}15}%
\textbf{,}%
\textbf{ Heure : }%
\textbf{15:09:54}%
\newline%
%
\textbf{P1{-}89 }%
\textbf{ : }%
\textbf{ Explosion de Tâche}%
\newline%
\newline%
%
Explosion des tâches matériels et de la communication avec Thomas Roy. Plus création d'un fichier excell pour ajout des risques.\newline%
\newline%
%
\newpage

%
\section{Logbook de Charles}%
\newpage

%
\section{Logbook de Cristhian}%
\newpage

%
\section{Logbook de Guillaume}%
\textbf{Date : }%
\textbf{2018{-}02{-}15}%
\textbf{,}%
\textbf{ Heure : }%
\textbf{13:16:43}%
\newline%
%
\textbf{P1{-}97 }%
\textbf{ : }%
\textbf{ Ajout des règles d'automation pour le logbook}%
\newline%
\newline%
%
Petit changement pour la gestion des commentaires\newline%
\newline%
%
\textbf{Date : }%
\textbf{2018{-}02{-}15}%
\textbf{,}%
\textbf{ Heure : }%
\textbf{22:20:57}%
\newline%
%
\textbf{P1{-}97 }%
\textbf{ : }%
\textbf{ Ajout des règles d'automation pour le logbook}%
\newline%
\newline%
%
Log book configurer de manière automatique et exportation en latex testé et fonctionnelle!~\newline%
~\newline%
\newline%
%
\textbf{Date : }%
\textbf{2018{-}02{-}21}%
\textbf{,}%
\textbf{ Heure : }%
\textbf{12:34:48}%
\newline%
%
\textbf{P1{-}133 }%
\textbf{ : }%
\textbf{ 8.6 Construction de la banque de filtre Mel et concevoir sa fonction}%
\newline%
\newline%
%
Ajout de la fonction~void mfcc\_melFilterBank\_create() dans CCS à partir de la fonction en matlab\newline%
\newline%
%
\textbf{Date : }%
\textbf{2018{-}02{-}25}%
\textbf{,}%
\textbf{ Heure : }%
\textbf{17:00:38}%
\newline%
%
\textbf{P1{-}106 }%
\textbf{ : }%
\textbf{ 3.1 Obtenir une application compatible UART}%
\newline%
\newline%
%
Conception d'une connection virtual Serie fonctionnel sur Python.\newline%
L'application est en mesure de lire et d'écrire sur les deux ports de la connection virtuelle et ces actions sont éffectuées à partir du GUI.~\newline%
\newline%
%
\textbf{Date : }%
\textbf{2018{-}02{-}26}%
\textbf{,}%
\textbf{ Heure : }%
\textbf{15:08:28}%
\newline%
%
\textbf{P1{-}111 }%
\textbf{ : }%
\textbf{ 3.2 Recherche du protocole à utiliser pour la communication UART}%
\newline%
\newline%
%
Ajout des fonctions global directement dans la classe application() et placement des widgets dans la grille pour un meilleur fini\newline%
\newline%
%
\textbf{Date : }%
\textbf{2018{-}02{-}26}%
\textbf{,}%
\textbf{ Heure : }%
\textbf{15:59:25}%
\newline%
%
\textbf{P1{-}132 }%
\textbf{ : }%
\textbf{ 8.5 Réaliser la fonction FFT et en extraire le spectre de puissance}%
\newline%
\newline%
%
Ajout de la fonction powerSpectrum()\newline%
\newline%
%
\newpage

%
\section{Logbook de Jeffrey F.}%
\textbf{Date : }%
\textbf{2018{-}02{-}26}%
\textbf{,}%
\textbf{ Heure : }%
\textbf{12:35:07}%
\newline%
%
\textbf{P1{-}98 }%
\textbf{ : }%
\textbf{ 9.1 {-} Implémentation de l'auto{-}corrélation sur MATLAB}%
\newline%
\newline%
%
L'autocorrélation a étée faite sur MATLAB.\newline%
\newline%
%
\textbf{Date : }%
\textbf{2018{-}02{-}26}%
\textbf{,}%
\textbf{ Heure : }%
\textbf{12:38:23}%
\newline%
%
\textbf{P1{-}99 }%
\textbf{ : }%
\textbf{ 9.2 {-} Implémentation de l'auto{-}corrélation en C}%
\newline%
\newline%
%
L'autocorrélation a été implémentée en C.\newline%
\newline%
%
\textbf{Date : }%
\textbf{2018{-}02{-}26}%
\textbf{,}%
\textbf{ Heure : }%
\textbf{12:35:07}%
\newline%
%
\textbf{P1{-}98 }%
\textbf{ : }%
\textbf{ 9.1 {-} Implémentation de l'auto{-}corrélation sur MATLAB}%
\newline%
\newline%
%
L'autocorrélation a été implémentée sur MATLAB.\newline%
\newline%
%
\textbf{Date : }%
\textbf{2018{-}02{-}26}%
\textbf{,}%
\textbf{ Heure : }%
\textbf{12:40:06}%
\newline%
%
\textbf{P1{-}101 }%
\textbf{ : }%
\textbf{ 9.3 {-} Implémentation de l'auto{-}corrélation en ASM}%
\newline%
\newline%
%
Travail fait avec Jeffrey Rolland et Raphael Bouchard.\newline%
L'auto{-}corrélation a été implémentée en assembleur.\newline%
\newline%
%
\textbf{Date : }%
\textbf{2018{-}02{-}26}%
\textbf{,}%
\textbf{ Heure : }%
\textbf{12:40:06}%
\newline%
%
\textbf{P1{-}101 }%
\textbf{ : }%
\textbf{ 9.3 {-} Implémentation de l'auto{-}corrélation en ASM}%
\newline%
\newline%
%
25 février 2018.\newline%
Travail fait avec Jeffrey Rolland et Raphael Bouchard.\newline%
L'auto{-}corrélation a été implémentée en assembleur.\newline%
\newline%
%
\textbf{Date : }%
\textbf{2018{-}02{-}26}%
\textbf{,}%
\textbf{ Heure : }%
\textbf{12:38:23}%
\newline%
%
\textbf{P1{-}99 }%
\textbf{ : }%
\textbf{ 9.2 {-} Implémentation de l'auto{-}corrélation en C}%
\newline%
\newline%
%
25 février 2018.\newline%
Travail fait avec Jeffrey Rolland.\newline%
L'autocorrélation a été implémentée en C.\newline%
\newline%
%
\textbf{Date : }%
\textbf{2018{-}02{-}26}%
\textbf{,}%
\textbf{ Heure : }%
\textbf{12:35:07}%
\newline%
%
\textbf{P1{-}98 }%
\textbf{ : }%
\textbf{ 9.1 {-} Implémentation de l'auto{-}corrélation sur MATLAB}%
\newline%
\newline%
%
24 février 2018.\newline%
Travail fait avec Jeffrey Rolland.\newline%
L'autocorrélation a été implémentée sur MATLAB.\newline%
\newline%
%
\textbf{Date : }%
\textbf{2018{-}02{-}26}%
\textbf{,}%
\textbf{ Heure : }%
\textbf{12:38:23}%
\newline%
%
\textbf{P1{-}99 }%
\textbf{ : }%
\textbf{ 9.2 {-} Implémentation de l'auto{-}corrélation en C}%
\newline%
\newline%
%
24 février 2018.\newline%
Travail fait avec Jeffrey Rolland.\newline%
L'autocorrélation a été implémentée en C.\newline%
\newline%
%
\textbf{Date : }%
\textbf{2018{-}02{-}26}%
\textbf{,}%
\textbf{ Heure : }%
\textbf{12:44:03}%
\newline%
%
\textbf{P1{-}114 }%
\textbf{ : }%
\textbf{ Tests unitaires}%
\newline%
\newline%
%
26 février 2018.\newline%
Travail fait avec Raphael Bouchard.\newline%
Les tests pour l'auto{-}corrélation en C ont été faits.\newline%
\newline%
%
\textbf{Date : }%
\textbf{2018{-}02{-}26}%
\textbf{,}%
\textbf{ Heure : }%
\textbf{12:35:07}%
\newline%
%
\textbf{P1{-}98 }%
\textbf{ : }%
\textbf{ 9.1 {-} Implémentation de l'auto{-}corrélation sur MATLAB}%
\newline%
\newline%
%
24 février 2018.\newline%
Travail fait avec Jeffrey Rolland.\newline%
L'auto{-}corrélation a été implémentée sur MATLAB.\newline%
\newline%
%
\textbf{Date : }%
\textbf{2018{-}02{-}26}%
\textbf{,}%
\textbf{ Heure : }%
\textbf{12:38:23}%
\newline%
%
\textbf{P1{-}99 }%
\textbf{ : }%
\textbf{ 9.2 {-} Implémentation de l'auto{-}corrélation en C}%
\newline%
\newline%
%
24 février 2018.\newline%
Travail fait avec Jeffrey Rolland.\newline%
L'auto{-}corrélation a été implémentée en C.\newline%
\newline%
%
\textbf{Date : }%
\textbf{2018{-}02{-}26}%
\textbf{,}%
\textbf{ Heure : }%
\textbf{12:47:06}%
\newline%
%
\textbf{P1{-}116 }%
\textbf{ : }%
\textbf{ Tests unitaires}%
\newline%
\newline%
%
26 février 2018.\newline%
Travail fait avec Raphael Bouchard.\newline%
Les tests pour l'auto{-}corrélation en C ont été faits.\newline%
\newline%
%
\textbf{Date : }%
\textbf{2018{-}02{-}26}%
\textbf{,}%
\textbf{ Heure : }%
\textbf{12:47:50}%
\newline%
%
\textbf{P1{-}118 }%
\textbf{ : }%
\textbf{ Tests unitaires}%
\newline%
\newline%
%
26 février 2018.\newline%
Travail fait avec Raphael Bouchard.\newline%
Les tests pour l'auto{-}corrélation en assembleur ont été faits.\newline%
\newline%
%
\newpage

%
\section{Logbook de Jeffrey R.}%
\textbf{Date : }%
\textbf{2018{-}02{-}25}%
\textbf{,}%
\textbf{ Heure : }%
\textbf{19:17:03}%
\newline%
%
\textbf{P1{-}99 }%
\textbf{ : }%
\textbf{ 9.2 {-} Implémentation de l'auto{-}corrélation en C}%
\newline%
\newline%
%
24 Février 2018\newline%
Travail fait avec Jeffrey Fisher.\newline%
Implémentation de l'auto{-}corrélation en language C à partir de ce qui a été fait sur MATLAB complétée.\newline%
Fonctionne correctement selon nos tests préliminaires.\newline%
À faire : Plan de test et validation\newline%
\newline%
%
\textbf{Date : }%
\textbf{2018{-}02{-}25}%
\textbf{,}%
\textbf{ Heure : }%
\textbf{19:20:53}%
\newline%
%
\textbf{P1{-}101 }%
\textbf{ : }%
\textbf{ 9.3 {-} Implémentation de l'auto{-}corrélation en ASM}%
\newline%
\newline%
%
25 Février 2018\newline%
Travail fait avec Jeffrey Fisher et Raphael Bouchard.\newline%
Implémentation de l'auto{-}corrélation en language ASM à partir de ce qui a été fait en language C complétée.\newline%
Fonctionne correctement selon nos tests préliminaires.\newline%
À faire : Plan de test, validation et optimisation.\newline%
\newline%
%
\textbf{Date : }%
\textbf{2018{-}02{-}25}%
\textbf{,}%
\textbf{ Heure : }%
\textbf{19:25:32}%
\newline%
%
\textbf{P1{-}101 }%
\textbf{ : }%
\textbf{ 9.3 {-} Implémentation de l'auto{-}corrélation en ASM}%
\newline%
\newline%
%
25 Février 2018\newline%
Optimisation de l'algorithme d'auto{-}corrélation. Utilisation de parrallèlisme et retrait des temps d'attentes non{-}nécessaire au bon fonctionnement de l'algorithme.\newline%
Passé d'environ 3.1 million de cycles d'opération (environ 14ms) à environ 1.8 million cycle d'opération (8ms).\newline%
Fonctionne correctement selon nos tests préliminaires.\newline%
À faire : Plan de test et validation\newline%
\newline%
%
\textbf{Date : }%
\textbf{2018{-}02{-}26}%
\textbf{,}%
\textbf{ Heure : }%
\textbf{13:58:21}%
\newline%
%
\textbf{P1{-}101 }%
\textbf{ : }%
\textbf{ 9.3 {-} Implémentation de l'auto{-}corrélation en ASM}%
\newline%
\newline%
%
26 Février 2018\newline%
Optimisation de l'algorithme d'auto{-}corrélation. Logique de l'algorithme modifiée pour éviter de créer deux tableaux contenant des zeros. Nombre d'oppération mathématique reduite.\newline%
Réduction d'environ 800 000 cycle d'opération. Prend dorénavant environ 4,4ms.\newline%
\newline%
%
\newpage

%
\section{Logbook de Pierre{-}Yves}%
\textbf{Date : }%
\textbf{2018{-}02{-}15}%
\textbf{,}%
\textbf{ Heure : }%
\textbf{13:54:42}%
\newline%
%
\textbf{P1{-}11 }%
\textbf{ : }%
\textbf{ 1.1.1 {-} Caractérisation ADC}%
\newline%
\newline%
%
Résolution?\newline%
Plage d'entrée dynamique?\newline%
Offset?\newline%
\newline%
%
\textbf{Date : }%
\textbf{2018{-}02{-}15}%
\textbf{,}%
\textbf{ Heure : }%
\textbf{13:54:42}%
\newline%
%
\textbf{P1{-}11 }%
\textbf{ : }%
\textbf{ 1.1.1 {-} Caractérisation ADC}%
\newline%
\newline%
%
Résolution? 256x16bits\newline%
Plage d'entrée dynamique?\newline%
Offset?\newline%
\newline%
%
\textbf{Date : }%
\textbf{2018{-}02{-}15}%
\textbf{,}%
\textbf{ Heure : }%
\textbf{13:56:25}%
\newline%
%
\textbf{P1{-}23 }%
\textbf{ : }%
\textbf{ 1.1.1.1 {-} Paramétrage (configuration) ADC}%
\newline%
\newline%
%
DSK page20\newline%
\newline%
%
\textbf{Date : }%
\textbf{2018{-}02{-}25}%
\textbf{,}%
\textbf{ Heure : }%
\textbf{16:05:22}%
\newline%
%
\textbf{P1{-}23 }%
\textbf{ : }%
\textbf{ 1.2 {-} Paramétrage (configuration) ADC}%
\newline%
\newline%
%
Le fichier word en pièce jointe de cette tâche explique la configuration des registres utilisés par notre ADC\newline%
\newline%
%
\newpage

%
\section{Logbook de Pascal B.}%
\textbf{Date : }%
\textbf{2018{-}02{-}15}%
\textbf{,}%
\textbf{ Heure : }%
\textbf{15:37:36}%
\newline%
%
\textbf{P1{-}89 }%
\textbf{ : }%
\textbf{ Explosion de Tâche}%
\newline%
\newline%
%
Explosion des tâches de l'ADC avec P{-}Y\newline%
\newline%
%
\textbf{Date : }%
\textbf{2018{-}02{-}24}%
\textbf{,}%
\textbf{ Heure : }%
\textbf{15:46:49}%
\newline%
%
\textbf{P1{-}11 }%
\textbf{ : }%
\textbf{ 1.1.1 {-} Caractérisation ADC}%
\newline%
\newline%
%
{-}La fonction input\_left\_signal permet d'avoir un signal sonore déjà configuré. Il manquera a paramètré l'ADC pour qu'ils conviennent à nos besoins.\newline%
{-}Il a fallu rajouter une ligne de code dans l'Helper pour qu'il fonctionne (dans la fonction CODEC\_start, apres le set du sample rate pour set l'entrée)\newline%
\newline%
%
\textbf{Date : }%
\textbf{2018{-}02{-}24}%
\textbf{,}%
\textbf{ Heure : }%
\textbf{15:49:03}%
\newline%
%
\textbf{P1{-}13 }%
\textbf{ : }%
\textbf{ 1.1.2 {-} Conditionnement de signal ADC}%
\newline%
\newline%
%
Il ne semble pas avoir besoin de conditionnement de signal puisque des filtres actifs sont présents.\newline%
Tension d'offset déjà gerée\newline%
\newline%
%
\textbf{Date : }%
\textbf{2018{-}02{-}25}%
\textbf{,}%
\textbf{ Heure : }%
\textbf{17:41:26}%
\newline%
%
\textbf{P1{-}112 }%
\textbf{ : }%
\textbf{ 1.3 {-} Buffer données}%
\newline%
\newline%
%
{-}Buffer circulaire de données de 256 x short signé fonctionnel! Dans la SDRAM. Travaillé avec PY\newline%
\newline%
%
\newpage

%
\section{Logbook de Pascal L.}%
\textbf{Date : }%
\textbf{2018{-}02{-}16}%
\textbf{,}%
\textbf{ Heure : }%
\textbf{12:52:34}%
\newline%
%
\textbf{P1{-}122 }%
\textbf{ : }%
\textbf{ 8.1 Réaliser la structure général du code C du traitement MFCC}%
\newline%
\newline%
%
Création du projet CCS 7,\newline%
Création des fichiers source et auxiliaire, début des définitions des structures utiles\newline%
\newline%
%
\textbf{Date : }%
\textbf{2018{-}02{-}16}%
\textbf{,}%
\textbf{ Heure : }%
\textbf{12:55:23}%
\newline%
%
\textbf{P1{-}131 }%
\textbf{ : }%
\textbf{ 8.4 Réaliser la fonction du hamming window}%
\newline%
\newline%
%
Réalisation de la fonction hamming pour 256 valeurs flottantes,\newline%
Temps d'exécution :\newline%
version standard {-}\&gt;~ 0.035 ms\newline%
version optimisée {-}\&gt;~ 0.022 ms\newline%
\newline%
%
\textbf{Date : }%
\textbf{2018{-}02{-}17}%
\textbf{,}%
\textbf{ Heure : }%
\textbf{17:25:15}%
\newline%
%
\textbf{P1{-}122 }%
\textbf{ : }%
\textbf{ 8.1 Réaliser la structure général du code C du traitement MFCC}%
\newline%
\newline%
%
Ajout de prototypes de fonction dont la pre{-}emphasis, mel filter bank, ainsi que quelque définition de fonction annexe comme la conversion de float vers complexe\newline%
\newline%
%
\textbf{Date : }%
\textbf{2018{-}02{-}21}%
\textbf{,}%
\textbf{ Heure : }%
\textbf{12:36:19}%
\newline%
%
\textbf{P1{-}122 }%
\textbf{ : }%
\textbf{ 8.1 Réaliser la structure général du code C du traitement MFCC}%
\newline%
\newline%
%
Ajout des fonction de support pour la fft du DSP\newline%
\newline%
%
\textbf{Date : }%
\textbf{2018{-}02{-}21}%
\textbf{,}%
\textbf{ Heure : }%
\textbf{12:39:01}%
\newline%
%
\textbf{P1{-}132 }%
\textbf{ : }%
\textbf{ 8.5 Réaliser la fonction FFT et en extraire le spectre de puissance}%
\newline%
\newline%
%
Ajouts des fonctions de conversion float to complex,\newline%
ajout des fonctions utilitaires de la librairies DSPLIB associé à la fft single{-}precision\newline%
\newline%
%
\newpage

%
\section{Logbook de Raphael}%
\textbf{Date : }%
\textbf{2018{-}02{-}17}%
\textbf{,}%
\textbf{ Heure : }%
\textbf{16:19:23}%
\newline%
%
\textbf{P1{-}96 }%
\textbf{ : }%
\textbf{ Trouver la source des crash de Jira}%
\newline%
\newline%
%
Suite aux récents problèmes avec Jira, certains paramètres ont été modifié sur le serveur et il n'a pas l'air d'avoir recraché depuis.~\newline%
\newline%
%
\textbf{Date : }%
\textbf{2018{-}02{-}26}%
\textbf{,}%
\textbf{ Heure : }%
\textbf{09:45:53}%
\newline%
%
\textbf{P1{-}196 }%
\textbf{ : }%
\textbf{ Lecture sur la détection de silence}%
\newline%
\newline%
%
J'ai commencé à lire pour regarder les différentes approches afin de faire la détection de silence.\newline%
\newline%
%
\textbf{Date : }%
\textbf{2018{-}02{-}26}%
\textbf{,}%
\textbf{ Heure : }%
\textbf{12:30:21}%
\newline%
%
\textbf{P1{-}114 }%
\textbf{ : }%
\textbf{ Tests unitaires}%
\newline%
\newline%
%
Le plan de test pour l'autocorrélation sur Matlab est complété. Le document a été déposé dans le dossier \textbackslash{}\newline%
\newline%
%
\textbf{Date : }%
\textbf{2018{-}02{-}26}%
\textbf{,}%
\textbf{ Heure : }%
\textbf{12:31:54}%
\newline%
%
\textbf{P1{-}116 }%
\textbf{ : }%
\textbf{ Tests unitaires}%
\newline%
\newline%
%
Le plan de test pour l'autocorrélation en C est complété. Le document a été déposé dans le dossier \textbackslash{}\newline%
\newline%
%
\textbf{Date : }%
\textbf{2018{-}02{-}26}%
\textbf{,}%
\textbf{ Heure : }%
\textbf{12:32:17}%
\newline%
%
\textbf{P1{-}118 }%
\textbf{ : }%
\textbf{ 1h}%
\newline%
\newline%
%
Le plan de test pour l'autocorrélation en assembleur sur Code Composer Studio est complété. Le document a été déposé dans le dossier \textbackslash{}\newline%
\newline%
%
\newpage

%
\section{Logbook de Thomas}%
\textbf{Date : }%
\textbf{2018{-}02{-}16}%
\textbf{,}%
\textbf{ Heure : }%
\textbf{20:39:40}%
\newline%
%
\textbf{P1{-}89 }%
\textbf{ : }%
\textbf{ Explosion de Tâche}%
\newline%
\newline%
%
Test demandé par guillaume.. Major Tom to Ground Control\newline%
\newline%
%
\newpage

%
\end{document}