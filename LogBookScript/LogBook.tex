\documentclass{article}%
\usepackage[T1]{fontenc}%
\usepackage[utf8]{inputenc}%
\usepackage{lmodern}%
\usepackage{textcomp}%
\usepackage{lastpage}%
%
%
%
\begin{document}%
\normalsize%
\section{Logbook de Antoine}%
\textbf{Date : }%
\textbf{2018{-}02{-}15}%
\textbf{,}%
\textbf{ Heure : }%
\textbf{15:09:54}%
\newline%
%
\textbf{P1{-}89 }%
\textbf{ : }%
\textbf{ Explosion de Tâche}%
\newline%
\newline%
%
Explosion des tâches matériels et de la communication avec Thomas Roy. Plus création d'un fichier excell pour ajout des risques.\newline%
\newline%
%
\textbf{Date : }%
\textbf{2018{-}02{-}27}%
\textbf{,}%
\textbf{ Heure : }%
\textbf{22:32:32}%
\newline%
%
\textbf{P1{-}108 }%
\textbf{ : }%
\textbf{ 4.3 {-} Adapter le matériel initial pour répondre aux besoins du projet}%
\newline%
\newline%
%
J'ai dessouder la section \textbackslash{}\newline%
\newline%
%
\textbf{Date : }%
\textbf{2018{-}03{-}06}%
\textbf{,}%
\textbf{ Heure : }%
\textbf{17:16:03}%
\newline%
%
\textbf{P1{-}100 }%
\textbf{ : }%
\textbf{ 4.2 {-} Étude du matériel nécessaire pour le développement du projet}%
\newline%
\newline%
%
Définir le matériel nécessaire pour le projet et commander les pièces nécessaire (voir facture sur team). \newline%
\newline%
%
\textbf{Date : }%
\textbf{2018{-}03{-}06}%
\textbf{,}%
\textbf{ Heure : }%
\textbf{17:24:22}%
\newline%
%
\textbf{P1{-}200 }%
\textbf{ : }%
\textbf{ Mise à jour du schéma matériel}%
\newline%
\newline%
%
Mise à jour du PPt du schémas matériel sur le team.\newline%
\newline%
%
\textbf{Date : }%
\textbf{2018{-}03{-}08}%
\textbf{,}%
\textbf{ Heure : }%
\textbf{11:54:42}%
\newline%
%
\textbf{P1{-}201 }%
\textbf{ : }%
\textbf{ plan de test unitaire du matériel}%
\newline%
\newline%
%
J'ai créé le plan unitaire matériel qui comprend aussi les tests de communication. Il va rester à indiquer les résultats et la conclusion lorsqu'ils seront éffectués.\newline%
\newline%
%
\textbf{Date : }%
\textbf{2018{-}03{-}14}%
\textbf{,}%
\textbf{ Heure : }%
\textbf{19:51:33}%
\newline%
%
\textbf{P1{-}200 }%
\textbf{ : }%
\textbf{ Mise à jour du schéma matériel}%
\newline%
\newline%
%
Analyser les connexions exacte nécessaire et production du schémas électrique.\newline%
\newline%
%
\newpage

%
\section{Logbook de Charles}%
\newpage

%
\section{Logbook de Cristhian}%
\textbf{Date : }%
\textbf{2018{-}02{-}28}%
\textbf{,}%
\textbf{ Heure : }%
\textbf{14:54:23}%
\newline%
%
\textbf{P1{-}198 }%
\textbf{ : }%
\textbf{ 9.4 {-} Préparation de la démonstration \#1}%
\newline%
\newline%
%
Faire diagrammes UML pour l'auto{-}corrélation en C, MATLAB et ASM\newline%
\newline%
%
\newpage

%
\section{Logbook de Guillaume}%
\textbf{Date : }%
\textbf{2018{-}02{-}15}%
\textbf{,}%
\textbf{ Heure : }%
\textbf{13:16:43}%
\newline%
%
\textbf{P1{-}97 }%
\textbf{ : }%
\textbf{ Ajout des règles d'automation pour le logbook}%
\newline%
\newline%
%
Petit changement pour la gestion des commentaires\newline%
\newline%
%
\textbf{Date : }%
\textbf{2018{-}02{-}15}%
\textbf{,}%
\textbf{ Heure : }%
\textbf{22:20:57}%
\newline%
%
\textbf{P1{-}97 }%
\textbf{ : }%
\textbf{ Ajout des règles d'automation pour le logbook}%
\newline%
\newline%
%
Log book configurer de manière automatique et exportation en latex testé et fonctionnelle!~\newline%
~\newline%
\newline%
%
\textbf{Date : }%
\textbf{2018{-}02{-}21}%
\textbf{,}%
\textbf{ Heure : }%
\textbf{12:34:48}%
\newline%
%
\textbf{P1{-}133 }%
\textbf{ : }%
\textbf{ 8.6 Construction de la banque de filtre Mel et concevoir sa fonction}%
\newline%
\newline%
%
Ajout de la fonction~void mfcc\_melFilterBank\_create() dans CCS à partir de la fonction en matlab\newline%
\newline%
%
\textbf{Date : }%
\textbf{2018{-}02{-}25}%
\textbf{,}%
\textbf{ Heure : }%
\textbf{17:00:38}%
\newline%
%
\textbf{P1{-}106 }%
\textbf{ : }%
\textbf{ 3.1 Obtenir une application compatible UART}%
\newline%
\newline%
%
Conception d'une connection virtual Serie fonctionnel sur Python.\newline%
L'application est en mesure de lire et d'écrire sur les deux ports de la connection virtuelle et ces actions sont éffectuées à partir du GUI.~\newline%
\newline%
%
\textbf{Date : }%
\textbf{2018{-}02{-}26}%
\textbf{,}%
\textbf{ Heure : }%
\textbf{15:08:28}%
\newline%
%
\textbf{P1{-}111 }%
\textbf{ : }%
\textbf{ 3.2 Recherche du protocole à utiliser pour la communication UART}%
\newline%
\newline%
%
Ajout des fonctions global directement dans la classe application() et placement des widgets dans la grille pour un meilleur fini\newline%
\newline%
%
\textbf{Date : }%
\textbf{2018{-}02{-}26}%
\textbf{,}%
\textbf{ Heure : }%
\textbf{15:59:25}%
\newline%
%
\textbf{P1{-}132 }%
\textbf{ : }%
\textbf{ 8.5 Réaliser la fonction FFT et en extraire le spectre de puissance}%
\newline%
\newline%
%
Ajout de la fonction powerSpectrum()\newline%
\newline%
%
\textbf{Date : }%
\textbf{2018{-}02{-}27}%
\textbf{,}%
\textbf{ Heure : }%
\textbf{15:02:24}%
\newline%
%
\textbf{P1{-}122 }%
\textbf{ : }%
\textbf{ 8.1 Réaliser la structure général du code C du traitement MFCC}%
\newline%
\newline%
%
Ajout de la fonction pour calculer la moyenne glissante pour les silence\newline%
\newline%
%
\textbf{Date : }%
\textbf{2018{-}02{-}27}%
\textbf{,}%
\textbf{ Heure : }%
\textbf{15:05:52}%
\newline%
%
\textbf{P1{-}197 }%
\textbf{ : }%
\textbf{ 8.11 test bench MFCC}%
\newline%
\newline%
%
Création d'une fonction en C prenant en entrée un fichier csv et en sort les paramètres.~\newline%
\newline%
%
\textbf{Date : }%
\textbf{2018{-}02{-}27}%
\textbf{,}%
\textbf{ Heure : }%
\textbf{18:38:40}%
\newline%
%
\textbf{P1{-}197 }%
\textbf{ : }%
\textbf{ 8.11 test bench MFCC}%
\newline%
\newline%
%
Avancement difficile dû aux limitations du DSP. Avancement des testBenchs\newline%
\newline%
%
\textbf{Date : }%
\textbf{2018{-}03{-}01}%
\textbf{,}%
\textbf{ Heure : }%
\textbf{15:41:47}%
\newline%
%
\textbf{P1{-}197 }%
\textbf{ : }%
\textbf{ 8.11 test bench MFCC}%
\newline%
\newline%
%
Ajout du test benchpour tester la moyenne glissante. Reste à vérifier la nature de la petite différence entre résultats matlab et C.\newline%
Suspecte matlab\newline%
\newline%
%
\textbf{Date : }%
\textbf{2018{-}03{-}03}%
\textbf{,}%
\textbf{ Heure : }%
\textbf{01:14:43}%
\newline%
%
\textbf{P1{-}197 }%
\textbf{ : }%
\textbf{ 8.11 test bench MFCC}%
\newline%
\newline%
%
Moyenne glissante fonctionnelle, mais il y a une approximation dans le calcul qui faisait en sorte que les résultat n'était pas exactement les mêmes sur matlab et sur C. Une fois l'approximation implémenté en matlab, les résultats sont identiques. L'approximation de prendre la moyenne de 100 échantillons et de lui appliquer un facteur pour simuler une moyenne de 56 échantillons serait peut{-}être à ravoir si une plus grande précision est nécessaire.~\newline%
\newline%
%
\textbf{Date : }%
\textbf{2018{-}03{-}06}%
\textbf{,}%
\textbf{ Heure : }%
\textbf{14:32:43}%
\newline%
%
\textbf{P1{-}109 }%
\textbf{ : }%
\textbf{ 3.3 Gérer les interruptions externe }%
\newline%
\newline%
%
Implémentation des interruptions dans le GUI fonctionnelle. Je suis en mesure de lire de manière périodique sur le port UART et d'updater le texte affiché s'il y a eu une lecture non nulle. Ce n'est pas une interruption comme on le voit en assembleur, mais fait avec des threads qui se déclenche à l'aide d'un Timer.~\newline%
\newline%
%
\textbf{Date : }%
\textbf{2018{-}03{-}14}%
\textbf{,}%
\textbf{ Heure : }%
\textbf{21:50:08}%
\newline%
%
\textbf{P1{-}107 }%
\textbf{ : }%
\textbf{ 3.4 Concevoir un GUI alléchant}%
\newline%
\newline%
%
Travail sur la mise en place des widgets dans le GUI\newline%
~\newline%
\newline%
%
\textbf{Date : }%
\textbf{2018{-}03{-}14}%
\textbf{,}%
\textbf{ Heure : }%
\textbf{21:51:13}%
\newline%
%
\textbf{P1{-}208 }%
\textbf{ : }%
\textbf{ Revue 2 {-} Mise à jour des diagrammes UML}%
\newline%
\newline%
%
Schéma bloc et diagramme de classe de l'application python en plus d'un esquisse de la MEF\newline%
\newline%
%
\textbf{Date : }%
\textbf{2018{-}03{-}14}%
\textbf{,}%
\textbf{ Heure : }%
\textbf{21:52:20}%
\newline%
%
\textbf{P1{-}109 }%
\textbf{ : }%
\textbf{ 3.3 Gérer les interruptions externe }%
\newline%
\newline%
%
Optimisation de la classe de thread et ajout de deux fonctions pour démarrer et stopper les threads.\newline%
\newline%
%
\textbf{Date : }%
\textbf{2018{-}03{-}14}%
\textbf{,}%
\textbf{ Heure : }%
\textbf{23:58:28}%
\newline%
%
\textbf{P1{-}208 }%
\textbf{ : }%
\textbf{ Revue 2 {-} Mise à jour des diagrammes UML}%
\newline%
\newline%
%
Diagramme de classe retravaillé en ajoutant une description des fonctions\newline%
\newline%
%
\newpage

%
\section{Logbook de Jeffrey F.}%
\textbf{Date : }%
\textbf{2018{-}02{-}26}%
\textbf{,}%
\textbf{ Heure : }%
\textbf{12:40:06}%
\newline%
%
\textbf{P1{-}101 }%
\textbf{ : }%
\textbf{ 9.3 {-} Implémentation de l'auto{-}corrélation en ASM}%
\newline%
\newline%
%
25 février 2018.\newline%
Travail fait avec Jeffrey Rolland et Raphael Bouchard.\newline%
L'auto{-}corrélation a été implémentée en assembleur.\newline%
\newline%
%
\textbf{Date : }%
\textbf{2018{-}02{-}26}%
\textbf{,}%
\textbf{ Heure : }%
\textbf{12:44:03}%
\newline%
%
\textbf{P1{-}114 }%
\textbf{ : }%
\textbf{ Tests unitaires}%
\newline%
\newline%
%
26 février 2018.\newline%
Travail fait avec Raphael Bouchard.\newline%
Les tests pour l'auto{-}corrélation en C ont été faits.\newline%
\newline%
%
\textbf{Date : }%
\textbf{2018{-}02{-}26}%
\textbf{,}%
\textbf{ Heure : }%
\textbf{12:35:07}%
\newline%
%
\textbf{P1{-}98 }%
\textbf{ : }%
\textbf{ 9.1 {-} Implémentation de l'auto{-}corrélation sur MATLAB}%
\newline%
\newline%
%
24 février 2018.\newline%
Travail fait avec Jeffrey Rolland.\newline%
L'auto{-}corrélation a été implémentée sur MATLAB.\newline%
\newline%
%
\textbf{Date : }%
\textbf{2018{-}02{-}26}%
\textbf{,}%
\textbf{ Heure : }%
\textbf{12:38:23}%
\newline%
%
\textbf{P1{-}99 }%
\textbf{ : }%
\textbf{ 9.2 {-} Implémentation de l'auto{-}corrélation en C}%
\newline%
\newline%
%
24 février 2018.\newline%
Travail fait avec Jeffrey Rolland.\newline%
L'auto{-}corrélation a été implémentée en C.\newline%
\newline%
%
\textbf{Date : }%
\textbf{2018{-}02{-}26}%
\textbf{,}%
\textbf{ Heure : }%
\textbf{12:47:06}%
\newline%
%
\textbf{P1{-}116 }%
\textbf{ : }%
\textbf{ Tests unitaires}%
\newline%
\newline%
%
26 février 2018.\newline%
Travail fait avec Raphael Bouchard.\newline%
Les tests pour l'auto{-}corrélation en C ont été faits.\newline%
\newline%
%
\textbf{Date : }%
\textbf{2018{-}02{-}26}%
\textbf{,}%
\textbf{ Heure : }%
\textbf{12:47:50}%
\newline%
%
\textbf{P1{-}118 }%
\textbf{ : }%
\textbf{ Tests unitaires}%
\newline%
\newline%
%
26 février 2018.\newline%
Travail fait avec Raphael Bouchard.\newline%
Les tests pour l'auto{-}corrélation en assembleur ont été faits.\newline%
\newline%
%
\textbf{Date : }%
\textbf{2018{-}02{-}28}%
\textbf{,}%
\textbf{ Heure : }%
\textbf{18:33:14}%
\newline%
%
\textbf{P1{-}198 }%
\textbf{ : }%
\textbf{ 9.4 {-} Préparation de la démonstration \#1}%
\newline%
\newline%
%
Travail avec Jeffrey Rolland\newline%
Mise à jour finale pour l'autocorrélation afin de préparer pour la démonstration 1. Conception du schéma bloc.\newline%
\newline%
%
\textbf{Date : }%
\textbf{2018{-}03{-}15}%
\textbf{,}%
\textbf{ Heure : }%
\textbf{02:54:06}%
\newline%
%
\textbf{P1{-}210 }%
\textbf{ : }%
\textbf{ Revue 2 {-} Mise à jour de l'AQ}%
\newline%
\newline%
%
3h {-} Mise à jour du document de l'AQ, création du plan détaillé des niveaux 3 \&amp; 4, préparation du PowerPoint pour la revue 2.\newline%
\newline%
%
\newpage

%
\section{Logbook de Jeffrey R.}%
\textbf{Date : }%
\textbf{2018{-}02{-}25}%
\textbf{,}%
\textbf{ Heure : }%
\textbf{19:17:03}%
\newline%
%
\textbf{P1{-}99 }%
\textbf{ : }%
\textbf{ 9.2 {-} Implémentation de l'auto{-}corrélation en C}%
\newline%
\newline%
%
24 Février 2018\newline%
Travail fait avec Jeffrey Fisher.\newline%
Implémentation de l'auto{-}corrélation en language C à partir de ce qui a été fait sur MATLAB complétée.\newline%
Fonctionne correctement selon nos tests préliminaires.\newline%
À faire : Plan de test et validation\newline%
\newline%
%
\textbf{Date : }%
\textbf{2018{-}02{-}25}%
\textbf{,}%
\textbf{ Heure : }%
\textbf{19:20:53}%
\newline%
%
\textbf{P1{-}101 }%
\textbf{ : }%
\textbf{ 9.3 {-} Implémentation de l'auto{-}corrélation en ASM}%
\newline%
\newline%
%
25 Février 2018\newline%
Travail fait avec Jeffrey Fisher et Raphael Bouchard.\newline%
Implémentation de l'auto{-}corrélation en language ASM à partir de ce qui a été fait en language C complétée.\newline%
Fonctionne correctement selon nos tests préliminaires.\newline%
À faire : Plan de test, validation et optimisation.\newline%
\newline%
%
\textbf{Date : }%
\textbf{2018{-}02{-}25}%
\textbf{,}%
\textbf{ Heure : }%
\textbf{19:25:32}%
\newline%
%
\textbf{P1{-}101 }%
\textbf{ : }%
\textbf{ 9.3 {-} Implémentation de l'auto{-}corrélation en ASM}%
\newline%
\newline%
%
25 Février 2018\newline%
Optimisation de l'algorithme d'auto{-}corrélation. Utilisation de parrallèlisme et retrait des temps d'attentes non{-}nécessaire au bon fonctionnement de l'algorithme.\newline%
Passé d'environ 3.1 million de cycles d'opération (environ 14ms) à environ 1.8 million cycle d'opération (8ms).\newline%
Fonctionne correctement selon nos tests préliminaires.\newline%
À faire : Plan de test et validation\newline%
\newline%
%
\textbf{Date : }%
\textbf{2018{-}02{-}26}%
\textbf{,}%
\textbf{ Heure : }%
\textbf{13:58:21}%
\newline%
%
\textbf{P1{-}101 }%
\textbf{ : }%
\textbf{ 9.3 {-} Implémentation de l'auto{-}corrélation en ASM}%
\newline%
\newline%
%
26 Février 2018\newline%
Optimisation de l'algorithme d'auto{-}corrélation. Logique de l'algorithme modifiée pour éviter de créer deux tableaux contenant des zeros. Nombre d'oppération mathématique reduite.\newline%
Réduction d'environ 800 000 cycle d'opération. Prend dorénavant environ 4,4ms.\newline%
\newline%
%
\textbf{Date : }%
\textbf{2018{-}02{-}27}%
\textbf{,}%
\textbf{ Heure : }%
\textbf{15:25:06}%
\newline%
%
\textbf{P1{-}198 }%
\textbf{ : }%
\textbf{ 9.4 {-} Préparation de la démonstration \#1}%
\newline%
\newline%
%
Mise a jour finale des fonctions d'autocorrelation pour la demo. Le schema bloc pour la revue est droppé sur le team dans le folder demo 1.\newline%
\newline%
%
\textbf{Date : }%
\textbf{2018{-}03{-}15}%
\textbf{,}%
\textbf{ Heure : }%
\textbf{07:13:37}%
\newline%
%
\textbf{P1{-}64 }%
\textbf{ : }%
\textbf{ Revue 2 {-} Améliorer la gestion du temps}%
\newline%
\newline%
%
Courbe en S mise à jour pour la revue 2. Reste à mettre a jour le diagramme de gantt.\newline%
\newline%
%
\newpage

%
\section{Logbook de Pierre{-}Yves}%
\textbf{Date : }%
\textbf{2018{-}02{-}15}%
\textbf{,}%
\textbf{ Heure : }%
\textbf{13:54:42}%
\newline%
%
\textbf{P1{-}11 }%
\textbf{ : }%
\textbf{ 1.1.1 {-} Caractérisation ADC}%
\newline%
\newline%
%
Résolution?\newline%
Plage d'entrée dynamique?\newline%
Offset?\newline%
\newline%
%
\textbf{Date : }%
\textbf{2018{-}02{-}15}%
\textbf{,}%
\textbf{ Heure : }%
\textbf{13:54:42}%
\newline%
%
\textbf{P1{-}11 }%
\textbf{ : }%
\textbf{ 1.1.1 {-} Caractérisation ADC}%
\newline%
\newline%
%
Résolution? 256x16bits\newline%
Plage d'entrée dynamique?\newline%
Offset?\newline%
\newline%
%
\textbf{Date : }%
\textbf{2018{-}02{-}15}%
\textbf{,}%
\textbf{ Heure : }%
\textbf{13:56:25}%
\newline%
%
\textbf{P1{-}23 }%
\textbf{ : }%
\textbf{ 1.1.1.1 {-} Paramétrage (configuration) ADC}%
\newline%
\newline%
%
DSK page20\newline%
\newline%
%
\textbf{Date : }%
\textbf{2018{-}02{-}25}%
\textbf{,}%
\textbf{ Heure : }%
\textbf{16:05:22}%
\newline%
%
\textbf{P1{-}23 }%
\textbf{ : }%
\textbf{ 1.2 {-} Paramétrage (configuration) ADC}%
\newline%
\newline%
%
Le fichier word en pièce jointe de cette tâche explique la configuration des registres utilisés par notre ADC\newline%
\newline%
%
\textbf{Date : }%
\textbf{2018{-}02{-}27}%
\textbf{,}%
\textbf{ Heure : }%
\textbf{14:44:55}%
\newline%
%
\textbf{P1{-}112 }%
\textbf{ : }%
\textbf{ 1.3 {-} Buffer données}%
\newline%
\newline%
%
Écriture des plans de test et exécution de ceux{-}ci pour l'ADC. Il reste à les valider (Peer Review)\newline%
\newline%
%
\textbf{Date : }%
\textbf{2018{-}03{-}09}%
\textbf{,}%
\textbf{ Heure : }%
\textbf{16:07:43}%
\newline%
%
\textbf{P1{-}202 }%
\textbf{ : }%
\textbf{ Plan d'intégration}%
\newline%
\newline%
%
L'ordre de chaque module à intégrer chronologiquement a été établi\newline%
\newline%
%
\newpage

%
\section{Logbook de Pascal B.}%
\textbf{Date : }%
\textbf{2018{-}02{-}15}%
\textbf{,}%
\textbf{ Heure : }%
\textbf{15:37:36}%
\newline%
%
\textbf{P1{-}89 }%
\textbf{ : }%
\textbf{ Explosion de Tâche}%
\newline%
\newline%
%
Explosion des tâches de l'ADC avec P{-}Y\newline%
\newline%
%
\textbf{Date : }%
\textbf{2018{-}02{-}24}%
\textbf{,}%
\textbf{ Heure : }%
\textbf{15:46:49}%
\newline%
%
\textbf{P1{-}11 }%
\textbf{ : }%
\textbf{ 1.1.1 {-} Caractérisation ADC}%
\newline%
\newline%
%
{-}La fonction input\_left\_signal permet d'avoir un signal sonore déjà configuré. Il manquera a paramètré l'ADC pour qu'ils conviennent à nos besoins.\newline%
{-}Il a fallu rajouter une ligne de code dans l'Helper pour qu'il fonctionne (dans la fonction CODEC\_start, apres le set du sample rate pour set l'entrée)\newline%
\newline%
%
\textbf{Date : }%
\textbf{2018{-}02{-}24}%
\textbf{,}%
\textbf{ Heure : }%
\textbf{15:49:03}%
\newline%
%
\textbf{P1{-}13 }%
\textbf{ : }%
\textbf{ 1.1.2 {-} Conditionnement de signal ADC}%
\newline%
\newline%
%
Il ne semble pas avoir besoin de conditionnement de signal puisque des filtres actifs sont présents.\newline%
Tension d'offset déjà gerée\newline%
\newline%
%
\textbf{Date : }%
\textbf{2018{-}02{-}25}%
\textbf{,}%
\textbf{ Heure : }%
\textbf{17:41:26}%
\newline%
%
\textbf{P1{-}112 }%
\textbf{ : }%
\textbf{ 1.3 {-} Buffer données}%
\newline%
\newline%
%
{-}Buffer circulaire de données de 256 x short signé fonctionnel! Dans la SDRAM. Travaillé avec PY\newline%
\newline%
%
\newpage

%
\section{Logbook de Pascal L.}%
\textbf{Date : }%
\textbf{2018{-}02{-}16}%
\textbf{,}%
\textbf{ Heure : }%
\textbf{12:52:34}%
\newline%
%
\textbf{P1{-}122 }%
\textbf{ : }%
\textbf{ 8.1 Réaliser la structure général du code C du traitement MFCC}%
\newline%
\newline%
%
Création du projet CCS 7,\newline%
Création des fichiers source et auxiliaire, début des définitions des structures utiles\newline%
\newline%
%
\textbf{Date : }%
\textbf{2018{-}02{-}16}%
\textbf{,}%
\textbf{ Heure : }%
\textbf{12:55:23}%
\newline%
%
\textbf{P1{-}131 }%
\textbf{ : }%
\textbf{ 8.4 Réaliser la fonction du hamming window}%
\newline%
\newline%
%
Réalisation de la fonction hamming pour 256 valeurs flottantes,\newline%
Temps d'exécution :\newline%
version standard {-}\&gt;~ 0.035 ms\newline%
version optimisée {-}\&gt;~ 0.022 ms\newline%
\newline%
%
\textbf{Date : }%
\textbf{2018{-}02{-}17}%
\textbf{,}%
\textbf{ Heure : }%
\textbf{17:25:15}%
\newline%
%
\textbf{P1{-}122 }%
\textbf{ : }%
\textbf{ 8.1 Réaliser la structure général du code C du traitement MFCC}%
\newline%
\newline%
%
Ajout de prototypes de fonction dont la pre{-}emphasis, mel filter bank, ainsi que quelque définition de fonction annexe comme la conversion de float vers complexe\newline%
\newline%
%
\textbf{Date : }%
\textbf{2018{-}02{-}21}%
\textbf{,}%
\textbf{ Heure : }%
\textbf{12:36:19}%
\newline%
%
\textbf{P1{-}122 }%
\textbf{ : }%
\textbf{ 8.1 Réaliser la structure général du code C du traitement MFCC}%
\newline%
\newline%
%
Ajout des fonction de support pour la fft du DSP\newline%
\newline%
%
\textbf{Date : }%
\textbf{2018{-}02{-}21}%
\textbf{,}%
\textbf{ Heure : }%
\textbf{12:39:01}%
\newline%
%
\textbf{P1{-}132 }%
\textbf{ : }%
\textbf{ 8.5 Réaliser la fonction FFT et en extraire le spectre de puissance}%
\newline%
\newline%
%
Ajouts des fonctions de conversion float to complex,\newline%
ajout des fonctions utilitaires de la librairies DSPLIB associé à la fft single{-}precision\newline%
\newline%
%
\textbf{Date : }%
\textbf{2018{-}02{-}27}%
\textbf{,}%
\textbf{ Heure : }%
\textbf{15:05:38}%
\newline%
%
\textbf{P1{-}197 }%
\textbf{ : }%
\textbf{ 8.11 test bench MFCC}%
\newline%
\newline%
%
Réaliser les testbench en Matlab et la génération des fichiers CSV utiliser par l'implémentation C pour sa validation\newline%
\newline%
%
\textbf{Date : }%
\textbf{2018{-}02{-}27}%
\textbf{,}%
\textbf{ Heure : }%
\textbf{15:07:39}%
\newline%
%
\textbf{P1{-}121 }%
\textbf{ : }%
\textbf{ 7.1 {-} Implémentation de l'algorithme de la moyenne mobile}%
\newline%
\newline%
%
Implémentation de la fonction C permettant utiliser le buffer circulaire de l'ADC pour effectuer une moyenne mobile optimisé\newline%
~\newline%
\newline%
%
\textbf{Date : }%
\textbf{2018{-}02{-}27}%
\textbf{,}%
\textbf{ Heure : }%
\textbf{18:38:22}%
\newline%
%
\textbf{P1{-}197 }%
\textbf{ : }%
\textbf{ 8.11 test bench MFCC}%
\newline%
\newline%
%
Création des fonctions de lecture de fichier CSV généré par matlab et testbench de la fonction freq2Mel() implémenté\newline%
\newline%
%
\textbf{Date : }%
\textbf{2018{-}02{-}28}%
\textbf{,}%
\textbf{ Heure : }%
\textbf{19:18:05}%
\newline%
%
\textbf{P1{-}197 }%
\textbf{ : }%
\textbf{ 8.11 test bench MFCC}%
\newline%
\newline%
%
Formatage des test bench pour en facilité la création pour les futures tests. cela inclue la création automatique de fichier log contenant les résultats complet du test bench global\newline%
\newline%
%
\textbf{Date : }%
\textbf{2018{-}03{-}08}%
\textbf{,}%
\textbf{ Heure : }%
\textbf{15:58:40}%
\newline%
%
\textbf{P1{-}134 }%
\textbf{ : }%
\textbf{ 8.7 Discret Cosinus Transform}%
\newline%
\newline%
%
Ajout de la DCT dans le projet CCS, elle a été testé avec succès. \newline%
Note: j'ai trouvé une erreur de signe dans la forme Matlab et C, l'erreur n'a toutefois que de très léger impact sur la validité des calculs, donc les valeurs de performance calculé précédement sur matlab sont toujours adéquat\newline%
~\newline%
\newline%
%
\textbf{Date : }%
\textbf{2018{-}03{-}08}%
\textbf{,}%
\textbf{ Heure : }%
\textbf{16:01:42}%
\newline%
%
\textbf{P1{-}197 }%
\textbf{ : }%
\textbf{ 8.11 test bench MFCC}%
\newline%
\newline%
%
Ajout de la fonction DCT, getMelCoeff, MelFilter dans le testbench,\newline%
Elles passent tousses les marges d'erreurs\newline%
\newline%
%
\textbf{Date : }%
\textbf{2018{-}03{-}08}%
\textbf{,}%
\textbf{ Heure : }%
\textbf{16:03:08}%
\newline%
%
\textbf{P1{-}133 }%
\textbf{ : }%
\textbf{ 8.6 Construction de la banque de filtre Mel et concevoir sa fonction}%
\newline%
\newline%
%
Test unitaire de la fonction fait. Ajout de la fonction mfcc\_getMelCoeff afin d'extraire les coefficient à paritr du powerspectrum et de l;a banque de filtres\newline%
\newline%
%
\textbf{Date : }%
\textbf{2018{-}03{-}09}%
\textbf{,}%
\textbf{ Heure : }%
\textbf{13:14:50}%
\newline%
%
\textbf{P1{-}122 }%
\textbf{ : }%
\textbf{ 8.1 Réaliser la structure général du code C du traitement MFCC}%
\newline%
\newline%
%
Ajout d'une structure général pour le MFCC dans le testbench et début des fonctions de la pipeline final d'extraction de métriques\newline%
\newline%
%
\textbf{Date : }%
\textbf{2018{-}03{-}09}%
\textbf{,}%
\textbf{ Heure : }%
\textbf{15:29:55}%
\newline%
%
\textbf{P1{-}135 }%
\textbf{ : }%
\textbf{ 8.8 Stoker les métriques extraites}%
\newline%
\newline%
%
Le pipeline principale de l'MFCC est fonctionnel\newline%
~\newline%
\newline%
%
\textbf{Date : }%
\textbf{2018{-}03{-}12}%
\textbf{,}%
\textbf{ Heure : }%
\textbf{00:18:00}%
\newline%
%
\textbf{P1{-}204 }%
\textbf{ : }%
\textbf{ Diagramme des fonctions}%
\newline%
\newline%
%
Structure global détaillé presque fini\newline%
~\newline%
\newline%
%
\newpage

%
\section{Logbook de Raphael}%
\textbf{Date : }%
\textbf{2018{-}02{-}17}%
\textbf{,}%
\textbf{ Heure : }%
\textbf{16:19:23}%
\newline%
%
\textbf{P1{-}96 }%
\textbf{ : }%
\textbf{ Trouver la source des crash de Jira}%
\newline%
\newline%
%
Suite aux récents problèmes avec Jira, certains paramètres ont été modifié sur le serveur et il n'a pas l'air d'avoir recraché depuis.~\newline%
\newline%
%
\textbf{Date : }%
\textbf{2018{-}02{-}26}%
\textbf{,}%
\textbf{ Heure : }%
\textbf{09:45:53}%
\newline%
%
\textbf{P1{-}196 }%
\textbf{ : }%
\textbf{ Lecture sur la détection de silence}%
\newline%
\newline%
%
J'ai commencé à lire pour regarder les différentes approches afin de faire la détection de silence.\newline%
\newline%
%
\textbf{Date : }%
\textbf{2018{-}02{-}26}%
\textbf{,}%
\textbf{ Heure : }%
\textbf{12:30:21}%
\newline%
%
\textbf{P1{-}114 }%
\textbf{ : }%
\textbf{ Tests unitaires}%
\newline%
\newline%
%
Le plan de test pour l'autocorrélation sur Matlab est complété. Le document a été déposé dans le dossier \textbackslash{}\newline%
\newline%
%
\textbf{Date : }%
\textbf{2018{-}02{-}26}%
\textbf{,}%
\textbf{ Heure : }%
\textbf{12:31:54}%
\newline%
%
\textbf{P1{-}116 }%
\textbf{ : }%
\textbf{ Tests unitaires}%
\newline%
\newline%
%
Le plan de test pour l'autocorrélation en C est complété. Le document a été déposé dans le dossier \textbackslash{}\newline%
\newline%
%
\textbf{Date : }%
\textbf{2018{-}02{-}26}%
\textbf{,}%
\textbf{ Heure : }%
\textbf{12:32:17}%
\newline%
%
\textbf{P1{-}118 }%
\textbf{ : }%
\textbf{ 1h}%
\newline%
\newline%
%
Le plan de test pour l'autocorrélation en assembleur sur Code Composer Studio est complété. Le document a été déposé dans le dossier \textbackslash{}\newline%
\newline%
%
\textbf{Date : }%
\textbf{2018{-}03{-}15}%
\textbf{,}%
\textbf{ Heure : }%
\textbf{08:02:19}%
\newline%
%
\textbf{P1{-}210 }%
\textbf{ : }%
\textbf{ Revue 2 {-} Mise à jour de l'AQ}%
\newline%
\newline%
%
Le document d'asusurance a été mise à jour selon les quelques changements apportés depuis le début du projet. La nouvelle version est disponible sur le Microsoft Team. De plus, le plan d'assurance~ qualité niveaux 3 et 4~ a été complété pour la revue 2\newline%
~\newline%
\newline%
%
\newpage

%
\section{Logbook de Thomas}%
\textbf{Date : }%
\textbf{2018{-}02{-}16}%
\textbf{,}%
\textbf{ Heure : }%
\textbf{20:39:40}%
\newline%
%
\textbf{P1{-}89 }%
\textbf{ : }%
\textbf{ Explosion de Tâche}%
\newline%
\newline%
%
Test demandé par guillaume.. Major Tom to Ground Control\newline%
\newline%
%
\textbf{Date : }%
\textbf{2018{-}02{-}28}%
\textbf{,}%
\textbf{ Heure : }%
\textbf{20:35:28}%
\newline%
%
\textbf{P1{-}117 }%
\textbf{ : }%
\textbf{ Peer review}%
\newline%
\newline%
%
Tous les tests exécutés étaient pertinents. Les résultats semblent corrects. L'auto{-}corrélation est prête pour la démo 1.\newline%
\newline%
%
\textbf{Date : }%
\textbf{2018{-}03{-}06}%
\textbf{,}%
\textbf{ Heure : }%
\textbf{17:12:48}%
\newline%
%
\textbf{P1{-}100 }%
\textbf{ : }%
\textbf{ 4.2 {-} Étude du matériel nécessaire pour le développement du projet}%
\newline%
\newline%
%
Définir le matériel nécessaire pour le projet\newline%
Commande des pièces.\newline%
La facture est sur le team.\newline%
\newline%
%
\textbf{Date : }%
\textbf{2018{-}03{-}06}%
\textbf{,}%
\textbf{ Heure : }%
\textbf{17:24:02}%
\newline%
%
\textbf{P1{-}200 }%
\textbf{ : }%
\textbf{ Mise à jour du schéma matériel}%
\newline%
\newline%
%
Mise à jour du powerpoint du schéma matériel sur team.\newline%
\newline%
%
\newpage

%
\end{document}